\documentclass[11pt,twoside,a4paper]{article}
\usepackage{a4wide,amsmath,amssymb}
% \usepackage[ngerman]{babel}
\usepackage[utf8x]{inputenc}
\hyphenation{}

% Build pdf:
% $ pdflatex seminar_template.tex
% $ pdflatex seminar_template.tex
% Run pdflatex twice to get the references right...


%-------------------------- Formatting --------------------------%

% Picture and table title formats
\usepackage[bf,small]{caption}

%\usepackage{mathpazo}  % -- use Palatino --
%\usepackage{mathptmx}  % -- or Times --

% Use this to discern DRAFTs from final versions
%\usepackage{draftcopy}
%\draftcopySetGrey{0.90}   %   90% = very light grey
%\draftcopySetScale{1}

%--------------- line and paragraph distances ----------------------%
\setlength{\parindent}{0em}
\setlength{\parskip}{\medskipamount}    % distance between paragraphs

% correctly format URLs and email addresses
\usepackage{url}
% example for email addresses: \url{foo@bar.com}

% Tools for note taking, ideas...
\newcommand{\notesubsection}[1][unsorted idea]{%
\subsection*{#1}%
\addcontentsline{toc}{subsection}{#1}%
}

% side note:
\newcommand{\bemerkung}[1]{\marginpar{\small\textsl{\textsf{#1}}}}

% this adds a little "under construction" icon on the side
%
% For this to work you need an "Baustelle.eps" icon. Get it from here:
%  http://www.net.in.tum.de/teaching/WS04/routing/Baustelle.eps.gz
\newcommand{\baustelle}[1][]{
 \marginpar{%
   \centerline{\includegraphics[scale=0.3]{Baustelle.eps}}
   {\small\textsl{\textsf{\raggedright #1}}}
}}

\begin{document}

\title{Weird Machines}
\author{Federico Scrinzi \\
  (\texttt{federico.scrinzi@campus.tu-berlin.de})\\[5mm]
  "`Computer Security Seminar"' , \\
  Technische Universität Berlin
}
  
\date{WS\,2013/2014 (Version of \today)}

\maketitle

\abstract{Compressed version of content..}

\section{Introduction}
Here you shall give the motivation and structure of your paper. Answer the
questions: What is the problem? Why is it a problem? Why is it interesting to
solve the problem?

\section{Chapter}

\subsection{Subchapter}

\section{Chapter}

Shamir introduced \cite{sham79} the concept of ...

\section{Summary}
Summarize what our wrote, your findings ...


\begin{thebibliography}{12}
\bibitem[HaKT1 98]{HaKT1 98} \footnote{The bibliography should include all
used literature. All literature included in the bibliography should be quoted
at least once in your text.}
        Michael Harkavy, J. D. Tygar, Hiroaki Kikuchi: {\sl Multi-round 
        Anonymous Auction Protocols}; 1st IEEE Workshop on Dependable and 
        Real-Time E-Commerce Systems, 1998.
\bibitem[Sham\_79]{sham79}
        Adi Shamir: {\sl How to Share a Secret}; 
        Communications of the ACM 22/11 (1979), S. 612-613.
\end{thebibliography}
\end{document}

